\documentclass[pdflatex,compress,mathserif]{beamer}

%\usetheme[dark,framenumber,totalframenumber]{ElektroITK}
\usetheme[darktitle,framenumber,totalframenumber]{ElektroITK}

\usepackage[utf8]{inputenc}
\usepackage[T1]{fontenc}
\usepackage{lmodern}
\usepackage[bahasai]{babel}
\usepackage{amsmath}
\usepackage{amsfonts}
\usepackage{amssymb}
\usepackage{graphicx}
\usepackage{multicol}
\usepackage{lipsum}

\newcommand*{\Scale}[2][4]{\scalebox{#1}{$#2$}}%

\title{Sinyal dan Sistem}
\subtitle{Analisis Fourier untuk Sinyal dan Sistem Waktu Kontinyu}

\author{Mifta Nur Farid}

\begin{document}

\maketitle

\section{Pengantar}

\section{Respon Sistem LTI Waktu Kontinyu terhadap Eksponensial Kompleks}

\begin{frame}
	\begin{itemize}
		\item Pada sistem LTI, sinyal direpresentasikan sebagai kombinasi linier dari sinyal-sinyal dasar yang memiliki sifat:
		\begin{enumerate}
			\item Sekumpulan sinyal dasar dapat digunakan untuk membangun kelas sinyal yang luas dan berguna.
			\item Respons sistem LTI untuk setiap sinyal harus cukup sederhana dalam struktur untuk memberikan representasi yang sesuai untuk respons sistem terhadap sinyal apa pun yang dibangun sebagai kombinasi linier dari sinyal dasar.
		\end{enumerate}
	\end{itemize}
\end{frame}

\end{document}
