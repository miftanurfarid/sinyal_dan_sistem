\documentclass[pdflatex,compress,mathserif]{beamer}

%\usetheme[dark,framenumber,totalframenumber]{ElektroITK}
\usetheme[darktitle,framenumber,totalframenumber]{ElektroITK}

\usepackage[utf8]{inputenc}
\usepackage[T1]{fontenc}
\usepackage{lmodern}
\usepackage[bahasai]{babel}
\usepackage{amsmath}
\usepackage{amsfonts}
\usepackage{amssymb}
\usepackage{graphicx}
\usepackage{multicol}
\usepackage{lipsum}

\newcommand*{\Scale}[2][4]{\scalebox{#1}{$#2$}}%

\title{Sinyal dan Sistem}
\subtitle{Sinyal dan Sistem}

\author{Tim Dosen Pengampu}

\begin{document}

\maketitle

\section{Pengantar}

\section{Sinyal Waktu Kontinyu dan Sinyal Waktu Diskrit}

\section{Sinyal Eksponensial dan Sinyal Sinusoidal}

\section{Fungsi Unit Impulse dan Fungsi Unit Step}

\section{Sistem Waktu Kontinyu dan Sistem Waktu Diskrit}

\section{Sifat-Sifat Dasar Sistem}


\end{document}
